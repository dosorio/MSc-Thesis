\newpage{\pagestyle{empty}\cleardoublepage}
\newpage
\section*{Resumen}
\addcontentsline{toc}{chapter}{\numberline{}Abstract}
En este trabajo, se identificaron las proteínas y rutas metabólicas asociadas a la respuesta neuroprotectora mediada por el neuroesteroide sintético tibolona bajo un modelo inflamatorio inducido por palmitato usando análisis de balance de flujo (FBA). Para tal fin, se modelaron tres diferentes escenarios metabólicos (`saludable', `inflamado' y `medicado') bajo una reconstrucción metabólica tejido-especifica de astrocitos maduros construida a partir de datos de expresión génica. La reconstrucción metabólica se construyó, validó y limitó usando tres paquetes de software  (\texttt{`minval'}, \texttt{`g2f'} y \texttt{`exp2flux'}) liberados a través de los repositorios R CRAN durante el desarrollo de este trabajo. A partir de nuestros análisis, predecimos que la tibolona ejecuta sus acciones neuroprotectoras a través de la reducción de la neurotoxicidad mediada por el L-glutamato en astrocitos, induciendo la activación de varias rutas metabólicas con funciones neuroprotectoras asociadas como: Metabolismo de taurina, gluconeogénesis y rutas de señalización PPAR y mediadas por calcio. Adicional a esto, encontramos un aumento en la tasa de crecimiento asociado a la tibolona que podría estar relacionado con efectos secundarios reportados para los compuestos esteroideos en otros tipos celulares humanos.

\textbf{Palabras clave: Astrocitos, Tibolona, Neuroproteccion, Inflamación, Analisis de Balance de Flujo.}

\section*{Abstract}
In this work, proteins and metabolic pathways associated with the neuroprotective response mediated by the synthetic neurosteroid tibolone under a palmitate-induced inflammatory model were identified by flux balance analysis (FBA). Three different metabolic scenarios (`healthy', `inflamed' and `medicated') were modeled over a gene expression data-driven constructed tissue-specific metabolic reconstruction of mature astrocytes. Astrocyte reconstruction was built, validated and constrained using three open source software packages (\texttt{`minval'}, \texttt{`g2f'} and \texttt{`exp2flux'}) released through CRAN R repositories during the development of this work. From our analysis, we predict that tibolone execute their neuroprotective effects through a reduction of neurotoxicity mediated by L-glutamate in astrocytes, inducing the activation several metabolic pathways with neuroprotective actions associated such as taurine metabolism, gluconeogenesis, calcium and PPAR signaling pathways. Also, we found a tibolone associated increase in growth rate probably in concordance to previously reported side effects of steroid compounds in other human cell types.

\textbf{Keywords: Astrocytes, Tibolone, Neuroprotection, Inflammation, Flux Balance Analysis.}
\clearpage\null\newpage
