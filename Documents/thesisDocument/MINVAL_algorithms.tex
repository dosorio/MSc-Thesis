\chapter{MINVAL algorithms}
\renewcommand*{\algorithmcfname}{MINVAL algorithm}
\begin{algorithm}
\SetKwInOut{Input}{input}
\SetKw{create}{create}
\SetKw{verify}{verify}
\SetKw{add}{add}
\SetKwInOut{Output}{output}
\Input{A A set of stoichiometric reaction with the following format: \texttt{H2O[c] + Urea-1-carboxylate[c] <=>\ 2 CO2[c] + 2 NH3[c]}. Where arrows and plus signs are surrounded by a ``space character''. It is also expected that stoichiometry coefficients are surrounded by spaces. It also expects arrows to be in the form \texttt{=>} or \texttt{<=>}. Meaning that arrows like \texttt{==>}, \texttt{<==>}, \texttt{-->} or \texttt{->} will not be parsed and will lead to errors.}
\Output{A boolean value \texttt{TRUE} or \texttt{FALSE} for each stoichiometric reaction}
\BlankLine
\ForEach {stoichiometric reaction}{
\create {an empty boolean vector $V$}\;
\eIf{each metabolite has only just one coefficient}{\add{\texttt{TRUE}} \KwTo $V$}{\add{\texttt{FALSE}} \KwTo $V$}
\eIf{metabolites coefficients are not surrounded by parentheses}{\add{\texttt{TRUE}} \KwTo $V$}{\add{\texttt{FALSE}} \KwTo $V$}
\eIf{arrow symbol is between blank spaces}{\add{\texttt{TRUE}} \KwTo $V$}{\add{\texttt{FALSE}} \KwTo $V$}
\eIf{arrow symbol is \texttt{<=>} or \texttt{=>}}{\add{\texttt{TRUE}} \KwTo $V$}{\add{\texttt{FALSE}} \KwTo $V$}
\eIf{metabolites names are separated by a plus symbol \texttt{(+)} between blank spaces}{\add{\texttt{TRUE}} \KwTo $V$}{\add{\texttt{FALSE}} \KwTo $V$}
\eIf{substituents position are joined by an hyphen to the metabolite name}{\add{\texttt{TRUE}} \KwTo $V$}{\add{\texttt{FALSE}} \KwTo $V$}d
\eIf{all elements of $V$ are TRUE}{\KwRet TRUE}{\KwRet FALSE}
}
\caption{isValidSyntax}
\end{algorithm}
%%%%
%%%%
\begin{algorithm}
\SetKwInOut{Input}{input}
\SetKw{create}{create}
\SetKwFunction{extract}{extract}
\SetKw{add}{add}
\SetKwInOut{Output}{output}
\Input{\begin{description}
\item reactionList: A set of stoichiometric reaction with the following format: \texttt{H2O[c] + Urea-1-carboxylate[c] <=>\ 2 CO2[c] + 2 NH3[c]}. Where arrows and plus signs are surrounded by a ``space character''. It is also expected that stoichiometry coefficients are surrounded by spaces. It also expects arrows to be in the form \texttt{=>} or \texttt{<=>}. Meaning that arrows like \texttt{==>}, \texttt{<==>}, \texttt{-->} or \texttt{->} will not be parsed and will lead to errors.
\item referenceData: A chemical table containing data to evaluate the balance
\item ids: A mandatory id of metabolites id column in the referenceData
\item mFormula: An optional id of molecular formula column in the referenceData
\item mWeight: An optional id of molecular weight column in the referenceData
\item mCharge: An optional id of net charge column in the referenceData
\end{description} }
\Output{A boolean value \texttt{TRUE} or \texttt{FALSE} for each stoichiometric reaction}
\BlankLine
\ForEach{stoichiometric reaction}{
\extract{reactants}
}
\caption{isBalanced}
\end{algorithm}