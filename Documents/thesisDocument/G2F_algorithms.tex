\chapter{G2F algorithms}
\setcounter{algocf}{0}
\renewcommand*{\algorithmcfname}{G2F algorithm}
\begin{algorithm}
\SetKwInOut{Input}{input}
\SetKw{create}{create}
\SetKw{verify}{verify}
\SetKw{add}{add}
\SetKwInOut{Output}{output}
\Input{\begin{description}
\item reaction: A stoichiometric reaction with the following format: \texttt{H2O[c] + Urea-1-carboxylate[c] <=>\ 2 CO2[c] + 2 NH3[c]}. Where arrows and plus signs are surrounded by a ``space character''. It is also expected that stoichiometry coefficients are surrounded by spaces. It also expects arrows to be in the form \texttt{=>} or \texttt{<=>}. Meaning that arrows like \texttt{==>}, \texttt{<==>}, \texttt{-->} or \texttt{->} will not be parsed and will lead to errors.
\item reference: A set of stoichiometric reactions with the same format of reaction
\end{description}}
\Output{The addition cost of a stoichiometric reaction based in the metabolites that compound a reference}
\BlankLine
refMet $\leftarrow$ extract all metabolites from reference \tcc*{Applying regular expressions}
rxnMet $\leftarrow$ extract all metabolites from reaction \tcc*{Applying regular expressions}
\KwRet $\dfrac{\left(\mid\text{rxnMet}\mid-\mid\text{rxnMet}\in\text{refMet}\mid\right)}{\mid\text{rxnMet}\mid}$
\caption{additionCost}
\end{algorithm}
%%%%
%%%%
\begin{algorithm}
\SetKwInOut{Input}{input}
\SetKw{create}{create}
\SetKw{verify}{verify}
\SetKw{add}{add}
\SetKwInOut{Output}{output}
\Input{A valid model for the `\texttt{sybil}' R package. An object of class \texttt{modelorg}.}
\Output{A set of ID's associated to reactions without flux under all scenarios.}
\BlankLine
create a empty vector F\;
\ForEach{reaction in the model}{
set all objective coefficients as 0\;
assign selected reaction as objective function\;
optimize the model\;
identify reactions with flux different to 0\;
add IDs of identified reactions to F\;
}
\KwRet model reaction IDs $\not \in$ F
\caption{blockedReactions}
\end{algorithm}
%%%%
%%%%
\begin{algorithm}
\SetKwInOut{Input}{input}
\SetKw{create}{create}
\SetKw{verify}{verify}
\SetKw{add}{add}
\SetKwInOut{Output}{output}
\SetKwRepeat{Do}{do}{while}
\Input{\begin{description}
\item reactionList: A set of stoichiometric reactions with the following format: \texttt{H2O[c] + Urea-1-carboxylate[c] <=>\ 2 CO2[c] + 2 NH3[c]}. Where arrows and plus signs are surrounded by a ``space character''. It is also expected that stoichiometry coefficients are surrounded by spaces. It also expects arrows to be in the form \texttt{=>} or \texttt{<=>}. Meaning that arrows like \texttt{==>}, \texttt{<==>}, \texttt{-->} or \texttt{->} will not be parsed and will lead to errors.
\item reference: A set of stoichiometric reaction with the same format of \texttt{reactionList}
\item limit: An addition cost value to be used as a limit to select reactions to be added. Is calculated as $\dfrac{NumberNewMetabolites}{NumerOfMetabolites}$ for each reaction.
\item woCompartment: A boolean value \texttt{TRUE} to define if compartment labels should be removed of the \texttt{reactionList} stoichiometric reactions, \texttt{FALSE} is used as default.
\item consensus: A boolean value \texttt{TRUE} to define if reactionList and newReactions should be reported as a unique vector or \texttt{FALSE} if just newReactions should be reported.
\end{description}}
\Output{A set of stoichiometric reactions that fill the model gaps.}
\BlankLine
\caption{gapFill}
\If{woCompartment is \texttt{TRUE}}{remove compartments of \texttt{reactionList} and \texttt{reference} metabolites \tcc*{Applying regular expressions}}
extract orphans $\left(\text{orphanReactant} \cup \text{orphanProducts}\right)$\;
\Do{}{}
\eIf{consensus is \texttt{TRUE}}{\KwRet \texttt{reactionList} $\cup$ \texttt{newReactions}}{\KwRet \texttt{newReactions}}
\end{algorithm}