\chapter{G2F algorithms}
\setcounter{algocf}{0}
\renewcommand*{\algorithmcfname}{G2F algorithm}
\begin{algorithm}
\SetKwInOut{Input}{input}
\SetKw{create}{create}
\SetKw{verify}{verify}
\SetKw{add}{add}
\SetKwInOut{Output}{output}
\Input{\begin{description}
\item reaction: A stoichiometric reaction with the following format: \texttt{H2O[c] + Urea-1-carboxylate[c] <=>\ 2 CO2[c] + 2 NH3[c]}. Where arrows and plus signs are surrounded by a ``space character''. It is also expected that stoichiometry coefficients are surrounded by spaces. It also expects arrows to be in the form \texttt{=>} or \texttt{<=>}. Meaning that arrows like \texttt{==>}, \texttt{<==>}, \texttt{-->} or \texttt{->} will not be parsed and will lead to errors.
\item reference: A set of stoichiometric reactions with the same format of reaction
\end{description}}
\Output{The addition cost of a stoichiometric reaction based in the metabolites that compound a reference}
\BlankLine
refMet $\leftarrow$ extract all metabolites from reference \tcc*{Applying regular expressions}
rxnMet $\leftarrow$ extract all metabolites from reaction \tcc*{Applying regular expressions}
\KwRet $\dfrac{\left(\mid\text{rxnMet}\mid-\mid\text{rxnMet}\in\text{refMet}\mid\right)}{\mid\text{rxnMet}\mid}$
\caption{additionCost}
\end{algorithm}