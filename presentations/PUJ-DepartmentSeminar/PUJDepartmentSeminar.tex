\documentclass[10pt]{beamer}
\usetheme{m}
\usecolortheme[RGB={1,1,1}]{structure}
\usepackage{booktabs}
\usepackage[scale=2]{ccicons}
\usepackage{pgfplots}
\usepgfplotslibrary{dateplot}
\usepackage[utf8x]{inputenc}
\usepackage{ucs}
\usepackage[spanish]{babel}
\usepackage{amsmath}
\usepackage{amsfonts}
\usepackage{amssymb}
\usepackage{graphicx}
\title{Identificación de proteínas y rutas metabólicas asociadas a la respuesta neuroprotectora mediada por la tibolona en astrocitos bajo un modelo inflamatorio inducido.\vspace{-0.75\baselineskip}}
\date{Agosto 14, 2015}
\author{Daniel Camilo Osorio}
\institute{\textbf{Maestría en Bioinformática}\\ Universidad Nacional de Colombia\\\textbf{Laboratorio de Bioquímica Computacional y Bioinformática} \\ Pontificia Universidad Javeriana}
%\titlegraphic{\hfill\includegraphics[height=1.5cm]{imagenes/VUNAL}}
\begin{document}
\maketitle
\begin{frame}
\frametitle{Proteínas}
\begin{itemize}
\item Son moléculas formadas por cadenas lineales de aminoácidos.
\pause
\item Realizan funciones enzimáticas, estructurales y de transducción de señales entre otras.
\pause
\item Están determinadas mayoritariamente por la genética de los organismos.
\pause
\item El conjunto de las proteínas expresadas en una circunstancia determinada es denominado \emph{proteoma}.
\end{itemize}
\end{frame}
\begin{frame}
\frametitle{Proteoma}
\begin{itemize}
\item Es el equivalente proteínico del \emph{genoma}.
\item Es la totalidad de proteínas expresadas en una célula bajo ciertas condiciones ó etapa de desarrollo específicas.
\end{itemize}
\end{frame}
\begin{frame}
\frametitle{Proteómica}
\end{frame}
\begin{frame}
\frametitle{Métodos para caracterización de proteomas}
\end{frame}
\begin{frame}
\frametitle{Espectros de masas}
\end{frame}
\begin{frame}
\frametitle{Archivos *.MGF}
\texttt{
\\
TITLE=01-02.734.734.3 File:``01-02.RAW", NativeID:``controllerType=0 controllerNumber=1 scan=734"\\
BEGIN IONS\\
RTINSECONDS=810.6452\\
PEPMASS=423.252593994141 12337.3798828125\\
CHARGE=3+\\
129.1288300 52.872806549 \\
149.1461182 3.9003605843 \\
157.1478424 2.5976366997 \\
163.1104431 7.5093927383 \\
174.8226013 9.9194545746 \\
193.2301788 2.1630632877 \\
.\\
.\\
.\\
}
\end{frame}
\begin{frame}
\frametitle{Método para la identificación de proteínas}
\end{frame}
\begin{frame}
\frametitle{Motores de búsqueda}
\end{frame}
\begin{frame}
\frametitle{Selección de motores de búsqueda}
\end{frame}
\begin{frame}
\frametitle{Proteínas identificadas}
\end{frame}
\begin{frame}
\frametitle{Reconstrucciones metabólicas}
\begin{center}
\begin{tabular}{|c|c|c|}
\hline
\multicolumn{3}{|c|}{\textbf{REDES}}\\
\hline
\hline
REGULACIÓN&SEÑALIZACIÓN&METABOLISMO\\
\hline
\end{tabular}
\end{center}\end{frame}
\plain{¿preguntas?}
\end{document}
