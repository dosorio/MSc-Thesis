\documentclass[11pt,a4paper]{article}
\usepackage[utf8]{inputenc}
\usepackage[spanish]{babel}
\usepackage{amsmath}
\usepackage{amsfonts}
\usepackage{amssymb}
\usepackage[left=3cm,right=3cm,top=4cm,bottom=2cm]{geometry}
\begin{document}
\pagestyle{empty}
\begin{flushright}
Bucaramanga, Enero de 2017
\end{flushright}
Señores\\
\textbf{Dirección de Arrea Curricular Ingeniería de Sistemas e Industrial}\\
Facultad de Ingeniería\\
Universidad Nacional de Colombia\\
Bogotá\\
\\
Cordial saludo:\\
\\
Después de recibidas las sugerencias del Profesor Andrés González como jurado calificador de la Tesis de Maestría en Bioinformática, titulada: \textbf{“Identifying proteins and metabolic pathways associated with the neuroprotective response mediated by tibolone in astrocytes under an induced inflammatory model”}. Hago entrega de una nueva versión del documento implementando las siguientes sugerencias:
\begin{enumerate}
\item \emph{``Con el fin de entender de mejor manera como funciona la estrategia de curación recomiendo incorporar los algoritmos usados en los códigos desarrollados. Esto permite que un evaluador entienda lo que se está realizando y se facilite su comparación con otras estrategias reportadas en literatura.''} A sugerencia del evaluador, se adicionaron los algoritmos de las cuatro funciones de curación  (\texttt{`isValidSyntax'}, \texttt{`isBalanced'}, \texttt{`orphanReactants'} y \texttt{`orphanProducts'}) usadas e implementadas dentro del paquete \textbf{`MINVAL: MINimal VALidation for Stoichiometric Reactions'} como un capitulo anexo al documento original que inicia en la página No. 38.
\item \emph{``Presentar los algoritmos reportados para hacer curación: principalmente gapfill y gapfind y comparar el desempeño de estos con la nueva estrategia planteada.''} A sugerencia del evaluador, se adicionaron los algoritmos de las tres funciones de curación y llenado de gaps  (\texttt{`additionCost'}, \texttt{`gapFill'} y \texttt{`blockedReactions'}) usadas e implementadas dentro del paquete \textbf{G2F: Find and Fill Gaps in Metabolic Networks} como un capitulo anexo al documento original que inicia en la página No. 41.
\item \emph{``El la sección de integración de datos genómicos, nuevamente es muy complejo entender la operación del código sin algoritmo. Recomiendo citar y comparar los resultados con IMAT, GIMME, E-Flux y PROM, los cuales son muy usados para realizar tareas análogas.''} Se adicionaron los algoritmos de las funciones para la integración de datos genómicos y la comparación de modelos  (\texttt{`exp2flux'} y \texttt{`fluxDifferences'}) usadas e implementadas dentro del paquete \textbf{EXP2FLUX: Convert Gene EXPression Data to FBA FLUXes} como un capitulo anexo al documento original que inicia en la página No. 43.
\item \emph{``No me queda claro la racionalidad en la decisión de realizar un FVA después de integrar los datos.''}
\item \emph{``Uno de los procedimientos típicos antes de poner a prueba el modelo para plantear respuestas a diferentes escenarios metabólicos es la conciliación con datos reportados en literatura. Recomiendo que se realice esta etapa.''}
\item \emph{``Si existe una diferencia en el análisis de sensibilidad usado comúnmente en problemas de programación lineal a partir de precios sombra y lo que plantea el capítulo por favor mencionarlo. En caso contrario recomiendo usar los términos asociados a optimización para lograr un mejor entendimiento.''}
\end{enumerate}
Atentamente,\\
\\
\textbf{Daniel Camilo Osorio}\\
Estudiante Maestría en Bioinformática
\end{document}