\documentclass[11pt,a4paper]{article}
\usepackage[utf8]{inputenc}
\usepackage[spanish]{babel}
\usepackage{amsmath}
\usepackage{amsfonts}
\usepackage{amssymb}
\usepackage[left=3cm,right=3cm,top=4cm,bottom=2cm]{geometry}
\begin{document}
\pagestyle{empty}
\begin{flushright}
Bucaramanga, Enero de 2017
\end{flushright}
Señores\\
\textbf{Dirección de Arrea Curricular Ingeniería de Sistemas e Industrial}\\
Facultad de Ingeniería\\
Universidad Nacional de Colombia\\
Bogotá\\
\\
Cordial saludo:\\
\\
Después de recibidas las sugerencias del Profesor Andrés González como jurado calificador de la Tesis de Maestría en Bioinformática, titulada: \textbf{“Identifying proteins and metabolic pathways associated with the neuroprotective response mediated by tibolone in astrocytes under an induced inflammatory model”}. Hago entrega de una nueva versión del documento implementando las siguientes sugerencias:
\begin{enumerate}
\item A sugerencia del evaluador, se adicionaron los algoritmos de cuatro de las  funciones implementadas dentro del paquete \textbf{`MINVAL: MINimal VALidation for Stoichiometric Reactions'} utizadas en el desarrollo de este trabajo para realizar la curación de la sintaxis, el balance de masa/carga y  la identificación de metabolitos huerfanos (\texttt{`isValidSyntax'}, \texttt{`isBalanced'}, \texttt{`orphanReactants'} y \texttt{`orphanProducts'}) como un capitulo anexo al documento original que inicia en la página No. 38.

\item A sugerencia del evaluador, se adicionaron los algoritmos de tres de las funciones implementadas dentro del paquete \textbf{G2F: Find and Fill Gaps in Metabolic Networks} utilizadas para realizar la identificación de gaps, el llenado de los mismos, y la remoción de las reacciones bloqueadas (\texttt{`additionCost'}, \texttt{`gapFill'} y \texttt{`blockedReactions'}) como un capitulo anexo al documento original que inicia en la página No. 41. Tambien se adicionó una tabla comparativa entre las metodologias usadas por otros tres algoritmos implementados (\texttt{`fastGapFill'}, \texttt{`growMatch'} y \texttt{`SMILEY'}) y la nueva metodologia propuesta.
\item \emph{``El la sección de integración de datos genómicos, nuevamente es muy complejo entender la operación del código sin algoritmo. Recomiendo citar y comparar los resultados con IMAT, GIMME, E-Flux y PROM, los cuales son muy usados para realizar tareas análogas.''} Se adicionaron los algoritmos de las funciones para la integración de datos genómicos y la comparación de modelos  (\texttt{`exp2flux'} y \texttt{`fluxDifferences'}) usadas e implementadas dentro del paquete \textbf{EXP2FLUX: Convert Gene EXPression Data to FBA FLUXes} como un capitulo anexo al documento original que inicia en la página No. 43.

\item \emph{``No me queda claro la racionalidad en la decisión de realizar un FVA después de integrar los datos.''}

\item \emph{``Uno de los procedimientos típicos antes de poner a prueba el modelo para plantear respuestas a diferentes escenarios metabólicos es la conciliación con datos reportados en literatura. Recomiendo que se realice esta etapa.''}

\item \emph{``Si existe una diferencia en el análisis de sensibilidad usado comúnmente en problemas de programación lineal a partir de precios sombra y lo que plantea el capítulo por favor mencionarlo. En caso contrario recomiendo usar los términos asociados a optimización para lograr un mejor entendimiento.''}

\end{enumerate}
Atentamente,\\
\\
\textbf{Daniel Camilo Osorio}\\
Estudiante Maestría en Bioinformática
\end{document}