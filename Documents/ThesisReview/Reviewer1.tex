\documentclass[12pt,a4paper]{article}
\usepackage[utf8]{inputenc}
\usepackage[spanish]{babel}
\usepackage{amsmath}
\usepackage{amsfonts}
\usepackage{amssymb}
\usepackage[left=2cm,right=2cm,top=4cm,bottom=1cm]{geometry}
\begin{document}
\pagestyle{empty}
\begin{flushright}
Bucaramanga, Enero de 2017
\end{flushright}
Señores\\
\textbf{Dirección de Área Curricular Ingeniería de Sistemas e Industrial}\\
Facultad de Ingeniería\\
Universidad Nacional de Colombia\\
Bogotá\\
\\
Cordial saludo:\\
\\
Después de recibidas las sugerencias del Profesor Andrés González como jurado calificador de la Tesis de Maestría en Bioinformática, titulada: \textbf{“Identifying proteins and metabolic pathways associated with the neuroprotective response mediated by tibolone in astrocytes under an induced inflammatory model”}. Hago entrega de una nueva versión del documento implementando las siguientes sugerencias:
\begin{enumerate}
\item Se adicionaron los algoritmos de cuatro de las  funciones implementadas dentro del paquete \textbf{`MINVAL: MINimal VALidation for Stoichiometric Reactions'} utizadas en el desarrollo de este trabajo para realizar la curación de la sintaxis, el balance de masa/carga y  la identificación de metabolitos huerfanos (\texttt{`isValidSyntax'}, \texttt{`isBalanced'}, \texttt{`orphanReactants'} y \texttt{`orphanProducts'}) como un capítulo anexo al documento original que inicia en la página No. 40.

\item Se adicionaron los algoritmos de tres de las funciones implementadas dentro del paquete \textbf{`G2F: Find and Fill Gaps in Metabolic Networks'} utilizadas para realizar la identificación de gaps, el llenado de los mismos, y la remoción de las reacciones bloqueadas (\texttt{`additionCost'}, \texttt{`gapFill'} y \texttt{`blockedReactions'}) como un capítulo anexo al documento original que inicia en la página No. 43. También se adicionó una tabla (tabla 2-1) comparativa entre las metodologias usadas por otros cuatro algoritmos implementados (\texttt{`SMILEY'}, \texttt{`gap-Find/Fill'}, \texttt{`growMatch'} y \texttt{`fastGapFill'}) bajo diferentes entornos de desarrollo y una breve descripción de la nueva metodologia propuesta.

\item Se adicionaron los algoritmos de las funciones para la integración de datos genómicos y la comparación de modelos  (\texttt{`exp2flux'} y \texttt{`fluxDifferences'}) usadas e implementadas dentro del paquete \textbf{`EXP2FLUX: Convert Gene EXPression Data to FBA FLUXes'} como un capítulo anexo al documento original que inicia en la página No. 45. También se adicionó una tabla (tabla 3-1) comparativa entre las metodologías usadas por otros cuatro algoritmos (\texttt{`GIMME'}, \texttt{`iMAT'}, \texttt{`E-FLUX'} y \texttt{`PROM'}) implementados para la incorporación de datos de expresión en reconstrucciones metabolicas a escala genómica.

\item Se removió la mención al análisis de variabilidad de flujos (FVA) en el tercer capítulo del documento original. Dicho análisis se habia realizado para generar los puntos a ser graficados en la figura 3-1, y así mostrar el efecto que tiene la incorporación de los datos de expresión sobre el flujo de biomasa dentro del modelo metabolico. En su reemplazo, se añadió un gráfico que muestra la distribución de flujos asociados a la funcion objetivo definida para el metabolismo central de \textit{E. coli} en el modelo original y en el modelo con las restricciones añadidas por medio de la incorporación de los datos transcriptómicos.

\item Se removió la mención al análisis de sensibilidad en el cuarto capitulo del documento original debido a que su uso era incorrecto. El análisis aplicado sobre el modelo no fué un \emph{análisis de sensibilidad} puesto que no buscaba evaluar el impacto de la modificación de los metabolitos de entrada sobre las salidas del sistema, sino evaluar el efecto del bloqueo de una reacción directamente sobre el valor de optimización para la función objetivo; por tal motivo el nombre del método fue ajustado a \emph{knock-out de reaccion}.
\end{enumerate}

Con respecto a la solicitud de realizar la etapa de conciliación de los resultados con los datos reportados por la literatura, no realizamos ningun cambio sobre el documento, puesto que consideramos que dicha etapa se realizó durante el desarrollo del trabajo (lineas 809 - 824) con los datos especificos disponibles para astrocitos humanos. Estudios previos de astrocitos tales como: \emph{Cakir, T., Alsan, S., Saybaşili, H., Akin, A., \& Ulgen, K. O. (2006). Reconstruction and flux analysis of coupling between metabolic pathways of astrocytes and neurons: application to cerebral hypoxia. Theoretical biology \& medical modelling, 4, 48-48.}  y \emph{Sertbaş, M., Ülgen, K., \& Çakır, T. (2014). Systematic analysis of transcription‐level effects of neurodegenerative diseases on human brain metabolism by a newly reconstructed brain‐specific metabolic network. FEBS open bio, 4(1), 542-553} desarrollan dicha actividad usando información no especifica para astrocitos o provenientes de experimentos hechos en rata.\\


%\item  \emph{``Uno de los procedimientos típicos antes de poner a prueba el modelo para plantear respuestas a diferentes escenarios metabólicos es la conciliación con datos reportados en literatura. Recomiendo que se realice esta etapa.''}
\noindent Atentamente,\\
\\
\textbf{Daniel Camilo Osorio}\\
Estudiante Maestría en Bioinformática
\end{document}