\chapter{g2f: An R package to Find and Fill Gaps for genome-scale metabolic networks}
\textbf{Written by:} \textit{Kelly Botero, Daniel Osorio, Janneth Gonzalez and Andrés Pinzón-Velasco}.\\
Tissue-specific metabolic reconstructions are subsets of reactions highly associated to the metabolism of a specific tissue. They are constructed from measured expression data and permit characterize or predict the metabolic behavior for each tissue under any physiological condition. Due to only the reactions associated to an enzyme or gene can be mapped from the expression data, the spontaneous reactions and non facilitated diffusion are missing in first stages of a tissue-specific reconstruction. Those missing reactions create gaps that block the biomass flux inside the metabolic pathways. Gaps should be filled before model evaluation through Flux Balance Analysis (FBA) to obtain a biological response. The g2f package was designed as a tool to find the gaps (metabolites not produced or not consumed in any other reaction), and fill it from the stoichiometric reactions of a reference metabolic reconstruction using a weighting function. Also the option to download all the set of gene-associated stoichiometric reactions for a specific organism from the KEGG database is available.
\section{Introduction}
\section{Installation and functions}
\section{Summary}
\section{Bibliography}