\documentclass[12pt,spanish,fleqn,openany,letterpaper,pagesize]{scrbook}
\usepackage[ruled,linesnumbered,vlined]{algorithm2e}
\usepackage[utf8]{inputenc}
\usepackage[spanish]{babel}
\usepackage{amsmath}
\usepackage{amsfonts}
\usepackage{amssymb}
\usepackage{fancyhdr}
\usepackage{epsfig}
\usepackage{epic}
\usepackage{array}
\usepackage{eepic}
\usepackage{nccmath}
\usepackage{amsmath}
\usepackage{threeparttable}
\usepackage{amscd}
\usepackage{Sweave}
\usepackage{here}
\usepackage{graphicx}
\usepackage{blindtext}
\usepackage{lscape}
\usepackage{tabularx}
\usepackage{subfigure}
\usepackage{longtable}
\usepackage{lineno}
%\usepackage[sort&compress,round,comma]
\usepackage[sort&compress]{natbib}
%\linenumbers

\usepackage{rotating} %Para rotar texto, objetos y tablas seite. No se ve en DVI solo en PS. Seite 328 Hundebuch
                        %se usa junto con \rotate, \sidewidestable ....


\renewcommand{\theequation}{\thechapter-\arabic{equation}}
\renewcommand{\thefigure}{\textbf{\thechapter-\arabic{figure}}}
\renewcommand{\thetable}{\textbf{\thechapter-\arabic{table}}}


\pagestyle{fancyplain}%\addtolength{\headwidth}{\marginparwidth}
\textheight22.5cm \topmargin0cm \textwidth16.5cm
\oddsidemargin0.5cm \evensidemargin-0.5cm%
\renewcommand{\chaptermark}[1]{\markboth{\thechapter\; #1}{}}
\renewcommand{\sectionmark}[1]{\markright{\thesection\; #1}}
\lhead[\fancyplain{}{\thepage}]{\fancyplain{}{\rightmark}}
\rhead[\fancyplain{}{\leftmark}]{\fancyplain{}{\thepage}}
\fancyfoot{}
\thispagestyle{fancy}%


\addtolength{\headwidth}{0cm}
\unitlength1mm %Define la unidad LE para Figuras
\mathindent0cm %Define la distancia de las formulas al texto,  fleqn las descentra
\marginparwidth0cm
\parindent0cm %Define la distancia de la primera linea de un parrafo a la margen

%Para tablas,  redefine el backschlash en tablas donde se define la posici\'{o}n del texto en las
%casillas (con \centering \raggedright o \raggedleft)
\newcommand{\PreserveBackslash}[1]{\let\temp=\\#1\let\\=\temp}
\let\PBS=\PreserveBackslash

%Espacio entre lineas
\renewcommand{\baselinestretch}{1.1}

%Neuer Befehl f\"{u}r die Tabelle Eigenschaften der Aktivkohlen
\newcommand{\arr}[1]{\raisebox{1.5ex}[0cm][0cm]{#1}}

%Neue Kommandos
\usepackage{Befehle}
